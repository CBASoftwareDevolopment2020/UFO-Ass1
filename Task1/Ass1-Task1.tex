\documentclass[a4paper]{article}

\usepackage[ampersand]{easylist}
\usepackage[nottoc]{tocbibind}
\usepackage{hyperref}
\usepackage[utf8]{inputenc}
\usepackage[danish]{babel}

\bibliographystyle{plainnat}

\title{\textbf{Undersøgelse og formidling}\\ Aflevering 1}
\author{Nikolaj Nielsen}
\date{\today}

\begin{document}
\maketitle
\tableofcontents

\section*{}
Dette dokument har til formål at beskrive bachelorprojektets krav og formalia.

\section{Projekt}
Projektet skal dokumentere forståelse af og evne til at reflektere over profesionens praksis samt anvendelse af teori og metode i relation til en praksisnær problemstilling. Problemstillingen, der er central for uddannelsen, formuleres af den studerende, eventuelt i samarbejde med en virksomhed \cite[§ 16, stk. 2, nr. 3]{professionsbachelor}. Uddannelsesinstitutionen skal godkende problemstillingen \cite[sektion 5.4]{studieordning}.

\subsection{Formål}
Formålet med projektet er, at dokumentere evnen til på et analytisk og metodisk grundlag at kunne arbejde med en kompleks og praksisnær problemstilling i relation til en konkret opgave inden for IT-området \cite[sektion 5.4]{studieordning}.

\subsection{Læringsmål}
Projetet skal dokumentere at den studerende har opnået uddannelsens afgangsniveau \cite[sektion 5.4]{studieordning}.\\
\textbf{Viden}. Den uddannede har viden om:
\begin{itemize}
\item den strategiske rolle af test i systemudviking
\item globalisering af softwareproduktion
\item systemarkitektur og forståelse af dens strategiske betydning for virksomhedens forretning
\item anvendt teori og metode samt udbredte teknologier inden for domænet
\item forskellige databasetyper og deres anvendelse
\end{itemize}
\textbf{Færdigheder}. Den studerende kan:
\begin{itemize}
\item integrere it-systemer og udvikle systemer, som understøtter fremtidig integration
\item anvende kontrakter som en styrings- og koordineringsmekanisme i udviklingsprocessen
\item vurdere og vælge databasesystemer, samt designe, redesigne og driftsoptimere databaser
\item planlægge og styre udviklingsforløb med mange geografisk adskilte projektdeltagere
\item håndtere planlægning og gennemførelse af test af større it-systemer
\end{itemize}
\textbf{Kompetencer}. Den uddannede kan:
\begin{itemize}
\item identificere sammenhænge mellem anvendt teori, metode og teknologi og kan reflektere over disses egnethed i forskellige situationer
\item indgå professionelt i samarbejde omkring udvikling af store systemer ved anvendelse af udbredte metoder og teknologier
\item sætte sig ind i nye teknologier og standarder til håndtering af integration mellem systemer
\item gennem praksis udvikle egen kompetenceprofil fra primært at være en backend-udviklerprofil til at varetage opgaver som systemarkitekt
\item håndtere fastlæggelse og realisering af en såvel forretningsmæssig som teknologisk hensigtsmæssig arkitektur for store systemer
\end{itemize}

\section{Eksamen}
Den studerende kan vælge at arbejde individuelt. Uanset om den projektet er udarbejdet i en gruppe, kan den studerende vælge at blive eksamineret individuelt \cite[§ 13]{eksamen}.\\
Der skal foretages en individuel bedømmelse uagtet om man bliver eksamineret individuelt eller som gruppe \cite[§ 13, stk. 2]{eksamen}.\\\\
Eksamen består af et projekt og en mundtlig det, og der gives én samlet karakter. Ydermere kan eksamen ikke finde sted før uddannelses øvrige eksaminer er bestået \cite[§ 15, stk. 1, nr. 3]{eksamen}.

\subsection{Bedømmelse}
Grundlaget for bedømmelsen er den individuelle præstation \cite[§ 37]{eksamen}.\\
Ved bedømmelsen af projektet vil der udover det faglige indhold blive lagt vægt på formulerings- og staveevne, det faglige indhold skal dog vægtes tungest. Der kan gives dispensation for dette, hvis den studerende kan dokumentere relevant funktionsnedsættelse \cite[§ 37, stk. 4]{eksamen}.

\subsection{Karakter}
Den studerende bedømmes efter 7-trins-skalaen \cite[§ 1]{karakter}.
\begin{table}[!t]
\centering
\begin{tabular}{|p{.1\textwidth}|p{.1\textwidth}|p{.2\textwidth}|p{.6\textwidth}|}\hline
7-trins-skalaen&ETCS-skalaen&&\\\hline
12&A&For den fremragende præstation&"Karakteren 12 gives for den fremragende præstation, der demonstrerer udtømmende opfyldelse af fagets, fag- eller uddannelseselementets mål, med ingen eller få uvæsentlige mangler." \cite[§ 2]{karakter}\\\hline
10&B&For den fortrinlige præstation&"Karakteren 10 gives for den fortrinlige præstation, der demonstrerer omfattende opfyldelse af fagets, fag- eller uddannelseselementets mål, med nogle mindre væsentlige mangler." \cite[§ 3]{karakter}\\\hline
7&C&For den gode præstation&"Karakteren 7 gives for den gode præstation, der demonstrerer opfyldelse af fagets, fag- eller uddannelseselementets mål, med en del mangler." \cite[§ 4]{karakter}\\\hline
4&D&For den jævne præstation&"Karakteren 4 gives for den jævne præstation, der demonstrerer en mindre grad af opfyldelse af fagets, fag- eller uddannelseselementets mål, med adskillige væsentlige mangler." \cite[§ 5]{karakter}\\\hline
02&E&For den tilstrækkelige præstation&"Karakteren 02 gives for den tilstrækkelige præstation, der demonstrerer den minimalt acceptable grad af opfyldelse af fagets, fag- eller uddannelseselementets mål." \cite[§ 6]{karakter}\\\hline
00&Fx&For den utilstrækkelige præstation&"Karakteren 00 gives for den utilstrækkelige præstation, der ikke demonstrerer en acceptabel grad af opfyldelse af fagets, fag- eller uddannelseselementets mål." \cite[§ 7]{karakter}\\\hline
-3&F&For den ringe præstation&"Karakteren -3 gives for den helt uacceptable præstation." \cite[§ 8]{karakter}\\\hline
\end{tabular}
\caption{7-trins-skalaen}\label{tab:7-trins-skalaen}
\end{table}
\\For at se betydningen af de forskellige karakterer se Tabel \ref{tab:7-trins-skalaen} på side \pageref{tab:7-trins-skalaen}

\listoffigures
\listoftables
\bibliography{Ass1-Task1-References}

\end{document}
